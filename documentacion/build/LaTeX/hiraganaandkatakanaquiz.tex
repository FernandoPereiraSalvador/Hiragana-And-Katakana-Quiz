%% Generated by Sphinx.
\def\sphinxdocclass{report}
\documentclass[letterpaper,10pt,spanish]{sphinxmanual}
\ifdefined\pdfpxdimen
   \let\sphinxpxdimen\pdfpxdimen\else\newdimen\sphinxpxdimen
\fi \sphinxpxdimen=.75bp\relax
\ifdefined\pdfimageresolution
    \pdfimageresolution= \numexpr \dimexpr1in\relax/\sphinxpxdimen\relax
\fi
%% let collapsible pdf bookmarks panel have high depth per default
\PassOptionsToPackage{bookmarksdepth=5}{hyperref}

\PassOptionsToPackage{booktabs}{sphinx}
\PassOptionsToPackage{colorrows}{sphinx}

\PassOptionsToPackage{warn}{textcomp}
\usepackage[utf8]{inputenc}
\ifdefined\DeclareUnicodeCharacter
% support both utf8 and utf8x syntaxes
  \ifdefined\DeclareUnicodeCharacterAsOptional
    \def\sphinxDUC#1{\DeclareUnicodeCharacter{"#1}}
  \else
    \let\sphinxDUC\DeclareUnicodeCharacter
  \fi
  \sphinxDUC{00A0}{\nobreakspace}
  \sphinxDUC{2500}{\sphinxunichar{2500}}
  \sphinxDUC{2502}{\sphinxunichar{2502}}
  \sphinxDUC{2514}{\sphinxunichar{2514}}
  \sphinxDUC{251C}{\sphinxunichar{251C}}
  \sphinxDUC{2572}{\textbackslash}
\fi
\usepackage{cmap}
\usepackage[T1]{fontenc}
\usepackage{amsmath,amssymb,amstext}
\usepackage{babel}



\usepackage{tgtermes}
\usepackage{tgheros}
\renewcommand{\ttdefault}{txtt}



\usepackage[Sonny]{fncychap}
\ChNameVar{\Large\normalfont\sffamily}
\ChTitleVar{\Large\normalfont\sffamily}
\usepackage{sphinx}

\fvset{fontsize=auto}
\usepackage{geometry}


% Include hyperref last.
\usepackage{hyperref}
% Fix anchor placement for figures with captions.
\usepackage{hypcap}% it must be loaded after hyperref.
% Set up styles of URL: it should be placed after hyperref.
\urlstyle{same}

\addto\captionsspanish{\renewcommand{\contentsname}{Contents:}}

\usepackage{sphinxmessages}
\setcounter{tocdepth}{1}



\title{Hiragana and Katakana Quiz}
\date{30 de agosto de 2023}
\release{1.0}
\author{Fernando Pereira}
\newcommand{\sphinxlogo}{\vbox{}}
\renewcommand{\releasename}{Versión}
\makeindex
\begin{document}

\ifdefined\shorthandoff
  \ifnum\catcode`\=\string=\active\shorthandoff{=}\fi
  \ifnum\catcode`\"=\active\shorthandoff{"}\fi
\fi

\pagestyle{empty}
\sphinxmaketitle
\pagestyle{plain}
\sphinxtableofcontents
\pagestyle{normal}
\phantomsection\label{\detokenize{index::doc}}


\sphinxstepscope


\chapter{configuracion module}
\label{\detokenize{configuracion:module-configuracion}}\label{\detokenize{configuracion:configuracion-module}}\label{\detokenize{configuracion::doc}}\index{module@\spxentry{module}!configuracion@\spxentry{configuracion}}\index{configuracion@\spxentry{configuracion}!module@\spxentry{module}}\index{Configuracion (clase en configuracion)@\spxentry{Configuracion}\spxextra{clase en configuracion}}

\begin{fulllineitems}
\phantomsection\label{\detokenize{configuracion:configuracion.Configuracion}}
\pysigstartsignatures
\pysiglinewithargsret{\sphinxbfcode{\sphinxupquote{class\DUrole{w}{ }}}\sphinxcode{\sphinxupquote{configuracion.}}\sphinxbfcode{\sphinxupquote{Configuracion}}}{\sphinxparam{\DUrole{n}{root}}\sphinxparamcomma \sphinxparam{\DUrole{n}{menu\_principal}}}{}
\pysigstopsignatures
\sphinxAtStartPar
Bases: \sphinxcode{\sphinxupquote{object}}

\sphinxAtStartPar
Clase para gestionar la ventana de configuración.

\sphinxAtStartPar
Esta clase define una ventana de configuración que permite al usuario borrar
los datos de progreso del juego o abrir el repositorio en línea del proyecto.

\sphinxAtStartPar
Parámetros:
:param root: Objeto Tkinter representando la ventana principal.
:param menu\_principal: Objeto Tkinter de la ventana principal del programa.

\sphinxAtStartPar
Detalles:
La ventana contiene dos opciones: una para borrar los datos de progreso del jugador
y otra para abrir un enlace al repositorio de GitHub del proyecto.
\index{abrir\_repositorio() (método estático de configuracion.Configuracion)@\spxentry{abrir\_repositorio()}\spxextra{método estático de configuracion.Configuracion}}

\begin{fulllineitems}
\phantomsection\label{\detokenize{configuracion:configuracion.Configuracion.abrir_repositorio}}
\pysigstartsignatures
\pysiglinewithargsret{\sphinxbfcode{\sphinxupquote{static\DUrole{w}{ }}}\sphinxbfcode{\sphinxupquote{abrir\_repositorio}}}{}{}
\pysigstopsignatures
\sphinxAtStartPar
Abre un navegador web para mostrar el repositorio en línea.

\sphinxAtStartPar
Detalles:
Esta función utiliza la biblioteca \sphinxtitleref{webbrowser} para abrir un navegador web
y mostrar la URL proporcionada. En este caso, la URL es para un repositorio
en línea. La URL puede ser modificada según se desee.
\begin{quote}\begin{description}
\sphinxlineitem{Devuelve}
\sphinxAtStartPar
None

\end{description}\end{quote}

\end{fulllineitems}

\index{borrar\_progreso() (método estático de configuracion.Configuracion)@\spxentry{borrar\_progreso()}\spxextra{método estático de configuracion.Configuracion}}

\begin{fulllineitems}
\phantomsection\label{\detokenize{configuracion:configuracion.Configuracion.borrar_progreso}}
\pysigstartsignatures
\pysiglinewithargsret{\sphinxbfcode{\sphinxupquote{static\DUrole{w}{ }}}\sphinxbfcode{\sphinxupquote{borrar\_progreso}}}{}{}
\pysigstopsignatures
\sphinxAtStartPar
Borra los datos de progreso.

\sphinxAtStartPar
Detalles: Esta función crea una instancia de la clase \sphinxtitleref{GestionDatos} para borrar los datos de progreso
almacenados y llama a la función \sphinxtitleref{borrar\_datos}. Luego, muestra una ventana informativa para indicar que el
borrado se realizó exitosamente.
\begin{quote}\begin{description}
\sphinxlineitem{Devuelve}
\sphinxAtStartPar
None

\end{description}\end{quote}

\end{fulllineitems}

\index{cerrar\_ventana() (método de configuracion.Configuracion)@\spxentry{cerrar\_ventana()}\spxextra{método de configuracion.Configuracion}}

\begin{fulllineitems}
\phantomsection\label{\detokenize{configuracion:configuracion.Configuracion.cerrar_ventana}}
\pysigstartsignatures
\pysiglinewithargsret{\sphinxbfcode{\sphinxupquote{cerrar\_ventana}}}{}{}
\pysigstopsignatures
\sphinxAtStartPar
Cierra la ventana actual y restaura la ventana principal.

\sphinxAtStartPar
Detalles:
Esta función restaura la ventana principal del programa al invocar el método
\sphinxtitleref{deiconify()} en el objeto \sphinxtitleref{menu\_principal}. Luego, cierra la ventana actual
de configuración al invocar el método \sphinxtitleref{destroy()} en el objeto \sphinxtitleref{root}.
\begin{quote}\begin{description}
\sphinxlineitem{Devuelve}
\sphinxAtStartPar
None

\end{description}\end{quote}

\end{fulllineitems}

\index{mostrar\_confirmacion() (método de configuracion.Configuracion)@\spxentry{mostrar\_confirmacion()}\spxextra{método de configuracion.Configuracion}}

\begin{fulllineitems}
\phantomsection\label{\detokenize{configuracion:configuracion.Configuracion.mostrar_confirmacion}}
\pysigstartsignatures
\pysiglinewithargsret{\sphinxbfcode{\sphinxupquote{mostrar\_confirmacion}}}{}{}
\pysigstopsignatures
\sphinxAtStartPar
Muestra una ventana emergente de confirmación para borrar los datos de progreso.

\sphinxAtStartPar
Detalles:
Esta función muestra una ventana emergente de confirmación con un mensaje preguntando
al usuario si está seguro de borrar los datos de progreso. Si el usuario confirma,
se llama al método \sphinxtitleref{borrar\_progreso} para realizar la acción de borrado.
\begin{quote}\begin{description}
\sphinxlineitem{Devuelve}
\sphinxAtStartPar
None

\end{description}\end{quote}

\end{fulllineitems}


\end{fulllineitems}

\index{main() (en el módulo configuracion)@\spxentry{main()}\spxextra{en el módulo configuracion}}

\begin{fulllineitems}
\phantomsection\label{\detokenize{configuracion:configuracion.main}}
\pysigstartsignatures
\pysiglinewithargsret{\sphinxcode{\sphinxupquote{configuracion.}}\sphinxbfcode{\sphinxupquote{main}}}{\sphinxparam{\DUrole{n}{menu\_principal}}}{}
\pysigstopsignatures
\sphinxAtStartPar
Función principal para iniciar la ventana de configuración.

\sphinxAtStartPar
Esta función crea una nueva ventana secundaria (\sphinxtitleref{Toplevel}) utilizando
la instancia \sphinxtitleref{menu\_principal}. Luego, instancia la clase \sphinxtitleref{Configuracion}
para crear y mostrar la ventana de configuración en la nueva ventana.

\sphinxAtStartPar
Parámetros:
:param menu\_principal: Objeto Tkinter de la ventana principal del programa.
\begin{quote}\begin{description}
\sphinxlineitem{Devuelve}
\sphinxAtStartPar
None

\end{description}\end{quote}

\end{fulllineitems}


\sphinxstepscope


\chapter{custom\_message\_box module}
\label{\detokenize{custom_message_box:module-custom_message_box}}\label{\detokenize{custom_message_box:custom-message-box-module}}\label{\detokenize{custom_message_box::doc}}\index{module@\spxentry{module}!custom\_message\_box@\spxentry{custom\_message\_box}}\index{custom\_message\_box@\spxentry{custom\_message\_box}!module@\spxentry{module}}\index{custom\_message\_box() (en el módulo custom\_message\_box)@\spxentry{custom\_message\_box()}\spxextra{en el módulo custom\_message\_box}}

\begin{fulllineitems}
\phantomsection\label{\detokenize{custom_message_box:custom_message_box.custom_message_box}}
\pysigstartsignatures
\pysiglinewithargsret{\sphinxcode{\sphinxupquote{custom\_message\_box.}}\sphinxbfcode{\sphinxupquote{custom\_message\_box}}}{\sphinxparam{\DUrole{n}{title}}\sphinxparamcomma \sphinxparam{\DUrole{n}{message}}\sphinxparamcomma \sphinxparam{\DUrole{n}{font\_size}}}{}
\pysigstopsignatures
\sphinxAtStartPar
Crea una ventana emergente personalizada con un mensaje.

\sphinxAtStartPar
Esta función crea una nueva ventana emergente utilizando la biblioteca Tkinter,
mostrando un título y un mensaje proporcionados. El tamaño de fuente del mensaje
se puede personalizar con el parámetro \sphinxtitleref{font\_size}.

\sphinxAtStartPar
Parámetros:
:param title: El título de la ventana emergente.
:param message: El mensaje que se mostrará en la ventana.
:param font\_size: El tamaño de fuente del mensaje.

\sphinxAtStartPar
Detalles:
La función crea la ventana emergente con un título y un mensaje utilizando la
fuente «Helvetica» y el tamaño de fuente proporcionado. Luego, calcula la posición
para centrar la ventana en la pantalla y ajusta la geometría de la ventana en consecuencia.
\begin{quote}\begin{description}
\sphinxlineitem{Devuelve}
\sphinxAtStartPar
None

\end{description}\end{quote}

\end{fulllineitems}


\sphinxstepscope


\chapter{datos module}
\label{\detokenize{datos:module-datos}}\label{\detokenize{datos:datos-module}}\label{\detokenize{datos::doc}}\index{module@\spxentry{module}!datos@\spxentry{datos}}\index{datos@\spxentry{datos}!module@\spxentry{module}}\index{GestionDatos (clase en datos)@\spxentry{GestionDatos}\spxextra{clase en datos}}

\begin{fulllineitems}
\phantomsection\label{\detokenize{datos:datos.GestionDatos}}
\pysigstartsignatures
\pysigline{\sphinxbfcode{\sphinxupquote{class\DUrole{w}{ }}}\sphinxcode{\sphinxupquote{datos.}}\sphinxbfcode{\sphinxupquote{GestionDatos}}}
\pysigstopsignatures
\sphinxAtStartPar
Bases: \sphinxcode{\sphinxupquote{object}}

\sphinxAtStartPar
Clase para gestionar la recopilación, almacenamiento y visualización de datos de progreso del jugador.

\sphinxAtStartPar
Esta clase proporciona métodos para guardar y leer datos de partidas en un archivo JSON,
generar gráficos de progreso y mostrarlos en ventanas emergentes, y borrar los datos almacenados.

\sphinxAtStartPar
Atributos:
data\_file: Ruta al archivo JSON donde se almacenan los datos de las partidas.
\index{borrar\_datos() (método de datos.GestionDatos)@\spxentry{borrar\_datos()}\spxextra{método de datos.GestionDatos}}

\begin{fulllineitems}
\phantomsection\label{\detokenize{datos:datos.GestionDatos.borrar_datos}}
\pysigstartsignatures
\pysiglinewithargsret{\sphinxbfcode{\sphinxupquote{borrar\_datos}}}{}{}
\pysigstopsignatures
\sphinxAtStartPar
Borra todos los datos almacenados en el archivo de datos.

\sphinxAtStartPar
Esta función borra todos los datos almacenados en el archivo de datos, dejando el archivo en blanco.

\sphinxAtStartPar
Detalles:
La función utiliza un bloque de código “try…except” para manejar posibles errores durante la operación
de borrado de datos.

\sphinxAtStartPar
En el bloque “try”, se abre el archivo de datos en modo de escritura (“w”) y se escribe un corchete
cuadrado vacío (“{[}{]}”) en el archivo, lo que efectivamente borra todos los datos previamente almacenados.

\sphinxAtStartPar
Si se produce un error durante la operación, se muestra un mensaje de error emergente utilizando la función
“custom\_message\_box”, indicando el tipo de error que ocurrió.
\begin{quote}\begin{description}
\sphinxlineitem{Devuelve}
\sphinxAtStartPar
None

\end{description}\end{quote}

\end{fulllineitems}

\index{cerrar\_ventana\_registro() (método estático de datos.GestionDatos)@\spxentry{cerrar\_ventana\_registro()}\spxextra{método estático de datos.GestionDatos}}

\begin{fulllineitems}
\phantomsection\label{\detokenize{datos:datos.GestionDatos.cerrar_ventana_registro}}
\pysigstartsignatures
\pysiglinewithargsret{\sphinxbfcode{\sphinxupquote{static\DUrole{w}{ }}}\sphinxbfcode{\sphinxupquote{cerrar\_ventana\_registro}}}{\sphinxparam{\DUrole{n}{ventana\_registro}}\sphinxparamcomma \sphinxparam{\DUrole{n}{ventana\_principal}}}{}
\pysigstopsignatures
\sphinxAtStartPar
Cierra la ventana de registro y restaura la ventana principal.

\sphinxAtStartPar
Esta función cierra la ventana de registro (ventana emergente) y restaura
la visibilidad de la ventana principal.

\sphinxAtStartPar
Parámetros:
:param ventana\_registro: Objeto Tkinter de la ventana de registro (ventana emergente).
:param ventana\_principal: Objeto Tkinter de la ventana principal del programa.

\sphinxAtStartPar
Detalles:
La función cierra la ventana de registro invocando el método \sphinxtitleref{destroy()} en
el objeto \sphinxtitleref{ventana\_registro}, lo que provoca su cierre.

\sphinxAtStartPar
Luego, se utiliza \sphinxtitleref{deiconify()} en el objeto \sphinxtitleref{ventana\_principal} para restaurar
la visibilidad de la ventana principal que pudo haber sido ocultada cuando se
abrió la ventana de registro.
\begin{quote}\begin{description}
\sphinxlineitem{Devuelve}
\sphinxAtStartPar
None

\end{description}\end{quote}

\end{fulllineitems}

\index{generar\_grafico() (método de datos.GestionDatos)@\spxentry{generar\_grafico()}\spxextra{método de datos.GestionDatos}}

\begin{fulllineitems}
\phantomsection\label{\detokenize{datos:datos.GestionDatos.generar_grafico}}
\pysigstartsignatures
\pysiglinewithargsret{\sphinxbfcode{\sphinxupquote{generar\_grafico}}}{}{}
\pysigstopsignatures
\sphinxAtStartPar
Genera y devuelve un gráfico de progreso del jugador.

\sphinxAtStartPar
Esta función utiliza los datos almacenados en el archivo JSON de datos para
crear un gráfico de líneas que muestra el progreso del jugador a lo largo
del tiempo, con los errores cometidos en el eje “y” y las fechas en el eje “x”.

\sphinxAtStartPar
Detalles:
La función lee los datos almacenados en el archivo JSON utilizando el método
\sphinxtitleref{leer\_datos}. Si no hay datos, la función devuelve \sphinxtitleref{None}. Luego, crea un DataFrame
de pandas con los datos leídos y convierte la columna “fecha” en un objeto de tipo
fecha. Los datos se dividen en dos conjuntos según el alfabeto elegido (hiragana o katakana).
Se crea un gráfico de líneas con los datos de errores para cada conjunto y se personaliza
con etiquetas, título y formato de fecha en el eje x. Finalmente, se ajusta la apariencia
de los ejes y se devuelve la figura del gráfico.
\begin{quote}\begin{description}
\sphinxlineitem{Devuelve}
\sphinxAtStartPar
Una figura de matplotlib que representa el gráfico de progreso del jugador.

\end{description}\end{quote}

\end{fulllineitems}

\index{guardar() (método de datos.GestionDatos)@\spxentry{guardar()}\spxextra{método de datos.GestionDatos}}

\begin{fulllineitems}
\phantomsection\label{\detokenize{datos:datos.GestionDatos.guardar}}
\pysigstartsignatures
\pysiglinewithargsret{\sphinxbfcode{\sphinxupquote{guardar}}}{\sphinxparam{\DUrole{n}{numero\_errores}}\sphinxparamcomma \sphinxparam{\DUrole{n}{errores}}\sphinxparamcomma \sphinxparam{\DUrole{n}{alfabeto\_elegido}}}{}
\pysigstopsignatures
\sphinxAtStartPar
Guarda los datos de una partida en el archivo JSON de datos.

\sphinxAtStartPar
Esta función almacena los datos de una partida en el archivo JSON designado,
incluyendo la fecha, el número de errores cometidos, los errores cometidos
durante la partida y el alfabeto elegido.

\sphinxAtStartPar
Parámetros:
:param numero\_errores: El número de errores cometidos durante la partida.
:param errores: Un diccionario que contiene los errores cometidos y sus valores.
:param alfabeto\_elegido: El alfabeto elegido para la partida (por ejemplo, hiragana o katakana).

\sphinxAtStartPar
Detalles:
La función crea un diccionario \sphinxtitleref{data} con los detalles de la partida, incluyendo la fecha
actual, el número de errores, los errores y el alfabeto elegido. Luego, intenta cargar
el contenido del archivo JSON existente. Si el archivo no existe, crea una lista vacía.
La partida actual se inserta en la primera posición de la lista. Finalmente, se guarda
la lista actualizada en el archivo JSON.
\begin{quote}\begin{description}
\sphinxlineitem{Devuelve}
\sphinxAtStartPar
None

\end{description}\end{quote}

\end{fulllineitems}

\index{leer\_datos() (método de datos.GestionDatos)@\spxentry{leer\_datos()}\spxextra{método de datos.GestionDatos}}

\begin{fulllineitems}
\phantomsection\label{\detokenize{datos:datos.GestionDatos.leer_datos}}
\pysigstartsignatures
\pysiglinewithargsret{\sphinxbfcode{\sphinxupquote{leer\_datos}}}{}{}
\pysigstopsignatures
\sphinxAtStartPar
Lee y devuelve los datos almacenados en el archivo JSON de datos.

\sphinxAtStartPar
Esta función lee los datos almacenados en el archivo JSON designado y devuelve
el contenido en forma de una lista de diccionarios. Cada diccionario representa
los detalles de una partida, incluyendo la fecha, el número de errores, los errores
cometidos y el alfabeto elegido.

\sphinxAtStartPar
Detalles:
La función intenta cargar el contenido del archivo JSON existente. Si el archivo
no existe, devuelve una lista vacía. En caso contrario, retorna el contenido
del archivo JSON, que consiste en una lista de diccionarios.
\begin{quote}\begin{description}
\sphinxlineitem{Devuelve}
\sphinxAtStartPar
Una lista de diccionarios representando los datos de las partidas almacenadas.

\end{description}\end{quote}

\end{fulllineitems}

\index{mostrar\_grafico\_tkinter() (método de datos.GestionDatos)@\spxentry{mostrar\_grafico\_tkinter()}\spxextra{método de datos.GestionDatos}}

\begin{fulllineitems}
\phantomsection\label{\detokenize{datos:datos.GestionDatos.mostrar_grafico_tkinter}}
\pysigstartsignatures
\pysiglinewithargsret{\sphinxbfcode{\sphinxupquote{mostrar\_grafico\_tkinter}}}{\sphinxparam{\DUrole{n}{menu\_principal}}}{}
\pysigstopsignatures
\sphinxAtStartPar
Muestra un gráfico de progreso del jugador en una ventana emergente.

\sphinxAtStartPar
Esta función genera un gráfico de progreso utilizando el método \sphinxtitleref{generar\_grafico}
y lo muestra en una ventana emergente de Tkinter.

\sphinxAtStartPar
Parámetros:
:param menu\_principal: Objeto Tkinter de la ventana principal del programa.

\sphinxAtStartPar
Detalles:
La función llama al método \sphinxtitleref{generar\_grafico} para obtener la figura del gráfico de progreso.
Si se obtiene una figura, se crea una nueva ventana secundaria utilizando \sphinxtitleref{Toplevel},
y se configura su título, tamaño y apariencia.

\sphinxAtStartPar
Luego, se crea un widget de lienzo (\sphinxtitleref{FigureCanvasTkAgg}) que contiene el gráfico y se lo
empaqueta en la ventana secundaria.

\sphinxAtStartPar
Se configura un manejo especial para el evento de cierre de ventana (\sphinxtitleref{WM\_DELETE\_WINDOW}),
de modo que cuando el usuario cierre la ventana emergente, se restaure la ventana
principal \sphinxtitleref{menu\_principal}.

\sphinxAtStartPar
Si no hay datos para generar un gráfico, se muestra un mensaje emergente utilizando la
función \sphinxtitleref{custom\_message\_box} para informar al usuario que no hay datos de progreso.
\begin{quote}\begin{description}
\sphinxlineitem{Devuelve}
\sphinxAtStartPar
None

\end{description}\end{quote}

\end{fulllineitems}


\end{fulllineitems}


\sphinxstepscope


\chapter{juego module}
\label{\detokenize{juego:module-juego}}\label{\detokenize{juego:juego-module}}\label{\detokenize{juego::doc}}\index{module@\spxentry{module}!juego@\spxentry{juego}}\index{juego@\spxentry{juego}!module@\spxentry{module}}\index{Juego (clase en juego)@\spxentry{Juego}\spxextra{clase en juego}}

\begin{fulllineitems}
\phantomsection\label{\detokenize{juego:juego.Juego}}
\pysigstartsignatures
\pysiglinewithargsret{\sphinxbfcode{\sphinxupquote{class\DUrole{w}{ }}}\sphinxcode{\sphinxupquote{juego.}}\sphinxbfcode{\sphinxupquote{Juego}}}{\sphinxparam{\DUrole{n}{menu\_principal}}}{}
\pysigstopsignatures
\sphinxAtStartPar
Bases: \sphinxcode{\sphinxupquote{object}}

\sphinxAtStartPar
Clase que representa el juego de aprendizaje de caracteres japoneses.

\sphinxAtStartPar
Esta clase implementa el juego de aprendizaje de caracteres japoneses. Los caracteres se presentan al jugador,
quien debe adivinar su significado en diferentes modos de juego: “japonés a español” o “español a japonés”.

\sphinxAtStartPar
Atributos:
\sphinxhyphen{} menu (None or Tkinter object): Almacena la ventana de juego en la que se mostrarán los caracteres y respuestas.
\sphinxhyphen{} numeroErrores (int): Contador del número de errores cometidos durante la partida actual.
\sphinxhyphen{} errores (dict): Diccionario que registra los errores cometidos, donde las claves son los caracteres japoneses
equivocados y los valores son sus significados.
\sphinxhyphen{} salida (bool): Indica si el juego ha terminado. Si es True, el juego continúa; si es False, el juego se detiene.
\sphinxhyphen{} menu\_principal (Tkinter object): Objeto Tkinter de la ventana principal del programa desde la cual se invocó el
juego.
\sphinxhyphen{} num\_letras\_conseguidas (int): Número de letras acertadas por el jugador en la partida actual, utilizado para
el seguimiento del progreso.

\sphinxAtStartPar
Métodos:
\sphinxhyphen{} \_\_init\_\_(self, menu\_principal): Inicializa una nueva instancia del juego.
\sphinxhyphen{} jugar(self, caracteres, modo\_juego, alfabeto\_elegido, menu\_principal): Inicia el juego de aprendizaje.
\sphinxhyphen{} japones\_a\_espanol(self, caracteres, alfabeto\_elegido, menu\_principal): Inicia el juego en el modo japonés a
español.
\sphinxhyphen{} espanol\_a\_japones(self, caracteres, alfabeto\_elegido, menu\_principal): Inicia el juego en el modo español a
japonés.
\sphinxhyphen{} generar\_opciones(caracteres\_copia, caracteres\_originales): Genera una lista de opciones posibles para el juego de
traducción.
\sphinxhyphen{} repetir(self, caracteres, alfabeto\_elegido, menu\_principal, modo, numero\_errores, errores): Muestra una ventana
emergente que pregunta al jugador si desea repetir el juego.
\sphinxhyphen{} repetir\_si(self, letras, alfabeto, menu\_principal, modo): Reinicia el juego para continuar jugando.
\sphinxhyphen{} repetir\_no(self, menu\_principal): Finaliza el juego y muestra el menú principal.
\sphinxhyphen{} cerrar\_ventana(self): Cierra la ventana de juego y restaura la ventana principal.
\sphinxhyphen{} crear\_contador(self, num\_letras\_conseguidas, num\_letras\_faltantes): Crea y muestra un contador de progreso en la
ventana de juego.
\sphinxhyphen{} crear\_ventana(self, num\_letras\_faltantes): Crea y muestra una nueva ventana de juego.
\index{cerrar\_ventana() (método de juego.Juego)@\spxentry{cerrar\_ventana()}\spxextra{método de juego.Juego}}

\begin{fulllineitems}
\phantomsection\label{\detokenize{juego:juego.Juego.cerrar_ventana}}
\pysigstartsignatures
\pysiglinewithargsret{\sphinxbfcode{\sphinxupquote{cerrar\_ventana}}}{}{}
\pysigstopsignatures
\sphinxAtStartPar
Cierra la ventana de juego y restaura la ventana principal.

\sphinxAtStartPar
Esta función se invoca cuando el jugador decide cerrar la ventana de juego en medio de una partida.
Cierra la ventana del juego actual, restaura la visibilidad de la ventana principal y detiene la partida.

\sphinxAtStartPar
Detalles:
La función establece el atributo \sphinxtitleref{salida} en \sphinxtitleref{False}, lo que detiene la partida actual. Luego, utiliza
los métodos \sphinxtitleref{withdraw()} y \sphinxtitleref{destroy()} en el objeto \sphinxtitleref{menu} para cerrar la ventana de juego actual.

\sphinxAtStartPar
Finalmente, utiliza \sphinxtitleref{deiconify()} en el objeto \sphinxtitleref{menu\_principal} para restaurar la visibilidad de la
ventana principal que pudo haber sido ocultada cuando se abrió la ventana de juego.
\begin{quote}\begin{description}
\sphinxlineitem{Devuelve}
\sphinxAtStartPar
None

\end{description}\end{quote}

\end{fulllineitems}

\index{crear\_contador() (método de juego.Juego)@\spxentry{crear\_contador()}\spxextra{método de juego.Juego}}

\begin{fulllineitems}
\phantomsection\label{\detokenize{juego:juego.Juego.crear_contador}}
\pysigstartsignatures
\pysiglinewithargsret{\sphinxbfcode{\sphinxupquote{crear\_contador}}}{\sphinxparam{\DUrole{n}{num\_letras\_conseguidas}}\sphinxparamcomma \sphinxparam{\DUrole{n}{num\_letras\_faltantes}}}{}
\pysigstopsignatures
\sphinxAtStartPar
Crea y muestra un contador de progreso en la ventana de juego.

\sphinxAtStartPar
Esta función crea y muestra un contador de progreso en la ventana de juego actual.
El contador muestra la cantidad de letras conseguidas y el total de letras faltantes
en el juego.

\sphinxAtStartPar
Parámetros:
:param num\_letras\_conseguidas: El número de letras conseguidas hasta el momento.
:param num\_letras\_faltantes: El número total de letras faltantes en el juego.

\sphinxAtStartPar
Detalles:
La función crea un marco (\sphinxtitleref{contador\_frame}) en la ventana de juego donde se mostrará
el contador. Luego, crea una etiqueta (\sphinxtitleref{contador\_label}) en el marco para mostrar
la información del contador en el formato «letras\_conseguidas / letras\_faltantes».

\sphinxAtStartPar
Los parámetros \sphinxtitleref{num\_letras\_conseguidas} y \sphinxtitleref{num\_letras\_faltantes} se utilizan para
mostrar los valores actuales en el contador.
\begin{quote}\begin{description}
\sphinxlineitem{Devuelve}
\sphinxAtStartPar
La etiqueta del contador creada.

\end{description}\end{quote}

\end{fulllineitems}

\index{crear\_ventana() (método de juego.Juego)@\spxentry{crear\_ventana()}\spxextra{método de juego.Juego}}

\begin{fulllineitems}
\phantomsection\label{\detokenize{juego:juego.Juego.crear_ventana}}
\pysigstartsignatures
\pysiglinewithargsret{\sphinxbfcode{\sphinxupquote{crear\_ventana}}}{\sphinxparam{\DUrole{n}{num\_letras\_faltantes}}}{}
\pysigstopsignatures
\sphinxAtStartPar
Crea y muestra una nueva ventana de juego.

\sphinxAtStartPar
Esta función crea y muestra una nueva ventana de juego utilizando la librería Tkinter.
La ventana muestra la interfaz gráfica del juego, incluyendo letras o caracteres
que el jugador debe identificar y una barra de progreso.

\sphinxAtStartPar
Parámetros:
:param num\_letras\_faltantes: El número total de letras faltantes en el juego.

\sphinxAtStartPar
Detalles:
La función utiliza la clase Toplevel para crear una nueva ventana secundaria que
contendrá la interfaz gráfica del juego. Se configuran varias propiedades de la ventana,
como su tamaño, posición, título y el ícono que se muestra en la barra de título.

\sphinxAtStartPar
Luego, se llama a la función \sphinxtitleref{crear\_contador} para crear y mostrar un contador de progreso
en la ventana de juego. El contador mostrará la cantidad de letras conseguidas y el total
de letras faltantes.
\begin{quote}\begin{description}
\sphinxlineitem{Devuelve}
\sphinxAtStartPar
None

\end{description}\end{quote}

\end{fulllineitems}

\index{espanol\_a\_japones() (método de juego.Juego)@\spxentry{espanol\_a\_japones()}\spxextra{método de juego.Juego}}

\begin{fulllineitems}
\phantomsection\label{\detokenize{juego:juego.Juego.espanol_a_japones}}
\pysigstartsignatures
\pysiglinewithargsret{\sphinxbfcode{\sphinxupquote{espanol\_a\_japones}}}{\sphinxparam{\DUrole{n}{caracteres}}\sphinxparamcomma \sphinxparam{\DUrole{n}{alfabeto\_elegido}}\sphinxparamcomma \sphinxparam{\DUrole{n}{menu\_principal}}}{}
\pysigstopsignatures
\sphinxAtStartPar
Inicia el modo de juego de traducción de español a japonés.

\sphinxAtStartPar
Esta función implementa el modo de juego en el que el jugador debe traducir palabras o frases del
español al japonés. Muestra una ventana con una palabra o frase en español y varias opciones en japonés,
de las cuales el jugador debe elegir la traducción correcta.

\sphinxAtStartPar
Parámetros:
:param caracteres: Un diccionario de palabras/frases en español y sus traducciones en japonés.
:param alfabeto\_elegido: El alfabeto elegido para el juego (hiragana o katakana).
:param menu\_principal: Objeto Tkinter de la ventana principal del programa.

\sphinxAtStartPar
Detalles:
La función inicializa variables y ajusta la configuración para el modo de juego.
Luego, inicia un ciclo en el cual se presentan palabras/frases en español junto con varias opciones
en japonés. El jugador debe seleccionar la opción correcta y se verifica su respuesta.

\sphinxAtStartPar
Si el jugador elige correctamente, se elimina la palabra/frase del diccionario de caracteres a adivinar
(\sphinxtitleref{caracteres}). Si elige incorrectamente, se muestra un mensaje de error con la respuesta correcta.

\sphinxAtStartPar
Al finalizar el juego, se guardan los datos de progreso y se ofrece al jugador la opción de repetir el juego.
\begin{quote}\begin{description}
\sphinxlineitem{Devuelve}
\sphinxAtStartPar
None

\end{description}\end{quote}

\end{fulllineitems}

\index{generar\_opciones() (método estático de juego.Juego)@\spxentry{generar\_opciones()}\spxextra{método estático de juego.Juego}}

\begin{fulllineitems}
\phantomsection\label{\detokenize{juego:juego.Juego.generar_opciones}}
\pysigstartsignatures
\pysiglinewithargsret{\sphinxbfcode{\sphinxupquote{static\DUrole{w}{ }}}\sphinxbfcode{\sphinxupquote{generar\_opciones}}}{\sphinxparam{\DUrole{n}{caracteres\_copia}}\sphinxparamcomma \sphinxparam{\DUrole{n}{caracteres\_originales}}}{}
\pysigstopsignatures
\sphinxAtStartPar
Genera una lista de opciones posibles para el juego de traducción.

\sphinxAtStartPar
Esta función crea una lista de opciones posibles para que el jugador elija durante el juego
de traducción de español a japonés. Estas opciones incluyen dos caracteres seleccionados al azar
de los caracteres copia y uno de los caracteres originales.

\sphinxAtStartPar
Parámetros:
:param caracteres\_copia: Un diccionario que contiene los caracteres disponibles para el juego.
:param caracteres\_originales: Un diccionario que contiene todos los caracteres originales.

\sphinxAtStartPar
Detalles:
La función elige al azar una opción original del diccionario de caracteres originales.
Luego, selecciona dos opciones al azar del diccionario de caracteres copia, asegurándose de que
sean diferentes de la opción original. Estas opciones se combinan en una lista llamada \sphinxtitleref{opciones\_copia}.

\sphinxAtStartPar
Finalmente, la opción original se agrega a la lista \sphinxtitleref{opciones\_copia}, y la lista resultante se
mezcla al azar utilizando \sphinxtitleref{random.shuffle()} para asegurar que las opciones no estén en un orden predecible.
\begin{quote}\begin{description}
\sphinxlineitem{Devuelve}
\sphinxAtStartPar
Una lista de opciones posibles para que el jugador elija durante el juego.

\end{description}\end{quote}

\end{fulllineitems}

\index{japones\_a\_espanol() (método de juego.Juego)@\spxentry{japones\_a\_espanol()}\spxextra{método de juego.Juego}}

\begin{fulllineitems}
\phantomsection\label{\detokenize{juego:juego.Juego.japones_a_espanol}}
\pysigstartsignatures
\pysiglinewithargsret{\sphinxbfcode{\sphinxupquote{japones\_a\_espanol}}}{\sphinxparam{\DUrole{n}{caracteres}}\sphinxparamcomma \sphinxparam{\DUrole{n}{alfabeto\_elegido}}\sphinxparamcomma \sphinxparam{\DUrole{n}{menu\_principal}}}{}
\pysigstopsignatures
\sphinxAtStartPar
Inicia el juego en el modo japonés a español.

\sphinxAtStartPar
Donde se muestran caracteres japoneses y el jugador debe adivinar su significado en español.

\sphinxAtStartPar
Parámetros:
:param caracteres: Un diccionario de caracteres japoneses como claves y sus significados como valores.
:param alfabeto\_elegido: El alfabeto elegido para el juego, como hiragana o katakana.
:param menu\_principal: Objeto Tkinter de la ventana principal del programa.

\sphinxAtStartPar
Detalles:
La función recibe el diccionario “caracteres” que contiene caracteres japoneses y sus significados
en español. “alfabeto\_elegido” especifica el alfabeto seleccionado para el juego, y “menu\_principal” es el
objeto Tkinter de la ventana principal del programa.

\sphinxAtStartPar
La función itera a través de los caracteres presentados en el diccionario “caracteres”, mostrando cada uno al
jugador. El jugador debe adivinar el significado en español del carácter japonés presentado. Si la respuesta
es correcta, el carácter se elimina del diccionario y el jugador avanza al siguiente. Si la respuesta es
incorrecta, se muestra un mensaje de error y se almacena la respuesta incorrecta en el diccionario de errores.

\sphinxAtStartPar
Luego de completar todos los caracteres o si el jugador decide salir, se guardan los datos de progreso
utilizando la instancia de la clase “GestionDatos” y se ofrece la opción de repetir el juego o volver al menú
principal.
\begin{quote}\begin{description}
\sphinxlineitem{Devuelve}
\sphinxAtStartPar
None

\end{description}\end{quote}

\end{fulllineitems}

\index{jugar() (método de juego.Juego)@\spxentry{jugar()}\spxextra{método de juego.Juego}}

\begin{fulllineitems}
\phantomsection\label{\detokenize{juego:juego.Juego.jugar}}
\pysigstartsignatures
\pysiglinewithargsret{\sphinxbfcode{\sphinxupquote{jugar}}}{\sphinxparam{\DUrole{n}{caracteres}}\sphinxparamcomma \sphinxparam{\DUrole{n}{modo\_juego}}\sphinxparamcomma \sphinxparam{\DUrole{n}{alfabeto\_elegido}}\sphinxparamcomma \sphinxparam{\DUrole{n}{menu\_principal}}}{}
\pysigstopsignatures
\sphinxAtStartPar
Inicia el juego de aprendizaje de caracteres japoneses.

\sphinxAtStartPar
Esta función inicia el juego de aprendizaje de caracteres japoneses. Los caracteres se presentan al jugador,
quien debe adivinar su significado cos dos modos: “modo japonés a español” o “modo español a japonés”.

\sphinxAtStartPar
Parámetros:
:param caracteres: Un diccionario de caracteres japoneses como claves y sus significados o
pronunciaciones como valores.
:param modo\_juego: Un valor booleano que indica si el juego se ejecuta en modo
japonés a español (True) o español a japonés (False).
:param alfabeto\_elegido: El alfabeto elegido para el
juego, como hiragana o katakana.
:param menu\_principal: Objeto Tkinter de la ventana principal del programa.

\sphinxAtStartPar
Detalles:
La función recibe el diccionario “caracteres” que contiene los caracteres japoneses y sus
significados o pronunciaciones. “modo\_juego” determina el modo de juego (japonés a español o español a
japonés). “alfabeto\_elegido” especifica el alfabeto seleccionado para el juego.

\sphinxAtStartPar
La función itera a través de los caracteres presentados en el diccionario “caracteres”, mostrando cada uno al
jugador. Dependiendo del modo de juego, llama a la función “japones\_a\_espanol” o “espanol\_a\_japones” para
manejar la interacción y la validación de respuestas.
\begin{quote}\begin{description}
\sphinxlineitem{Devuelve}
\sphinxAtStartPar
None

\end{description}\end{quote}

\end{fulllineitems}

\index{repetir() (método de juego.Juego)@\spxentry{repetir()}\spxextra{método de juego.Juego}}

\begin{fulllineitems}
\phantomsection\label{\detokenize{juego:juego.Juego.repetir}}
\pysigstartsignatures
\pysiglinewithargsret{\sphinxbfcode{\sphinxupquote{repetir}}}{\sphinxparam{\DUrole{n}{caracteres}}\sphinxparamcomma \sphinxparam{\DUrole{n}{alfabeto\_elegido}}\sphinxparamcomma \sphinxparam{\DUrole{n}{menu\_principal}}\sphinxparamcomma \sphinxparam{\DUrole{n}{modo}}\sphinxparamcomma \sphinxparam{\DUrole{n}{numero\_errores}}\sphinxparamcomma \sphinxparam{\DUrole{n}{errores}}}{}
\pysigstopsignatures
\sphinxAtStartPar
Muestra una ventana emergente que pregunta al jugador si desea repetir el juego.

\sphinxAtStartPar
Esta función crea una ventana emergente que pregunta al jugador si desea repetir el juego
con los mismos caracteres y configuración. El jugador puede elegir «Sí» o «No» en función
de su elección.

\sphinxAtStartPar
Parámetros:
:param caracteres: Un diccionario que contiene los caracteres para el juego.
:param alfabeto\_elegido: El alfabeto elegido para el juego (hiragana o katakana).
:param menu\_principal: Objeto Tkinter de la ventana principal del programa.
:param modo: Un valor booleano que indica el modo de juego (True para japonés a español, False para español a
japonés).
:param numero\_errores: El número de errores cometidos durante el juego.
:param errores: Un diccionario que contiene los errores cometidos y sus valores.

\sphinxAtStartPar
Detalles:
La función crea una nueva ventana emergente utilizando \sphinxtitleref{Toplevel}, con el título «¿Quieres repetir?»
y un tamaño fijo. Se agregan etiquetas para mostrar el texto «¿Quieres repetir?», el número de errores
cometidos y la fecha actual. Luego, se colocan dos botones «Sí» y «No» en el centro de la ventana.

\sphinxAtStartPar
Los botones «Sí» y «No» están asociados a las funciones \sphinxtitleref{repetir\_si()} y \sphinxtitleref{repetir\_no()} respectivamente,
que manejan la lógica de repetir el juego o regresar al menú principal.
\begin{quote}\begin{description}
\sphinxlineitem{Devuelve}
\sphinxAtStartPar
None

\end{description}\end{quote}

\end{fulllineitems}

\index{repetir\_no() (método de juego.Juego)@\spxentry{repetir\_no()}\spxextra{método de juego.Juego}}

\begin{fulllineitems}
\phantomsection\label{\detokenize{juego:juego.Juego.repetir_no}}
\pysigstartsignatures
\pysiglinewithargsret{\sphinxbfcode{\sphinxupquote{repetir\_no}}}{\sphinxparam{\DUrole{n}{menu\_principal}}}{}
\pysigstopsignatures
\sphinxAtStartPar
Finaliza el juego y muestra el menú principal.

\sphinxAtStartPar
Esta función finaliza el juego y muestra nuevamente el menú principal del programa.
Se invoca cuando el jugador decide no repetir el juego y regresar al menú principal.

\sphinxAtStartPar
Parámetros:
:param menu\_principal: Objeto Tkinter de la ventana principal del programa.

\sphinxAtStartPar
Detalles:
La función destruye la ventana actual del juego utilizando el método \sphinxtitleref{destroy()}, lo que
cierra la ventana emergente del juego actual.

\sphinxAtStartPar
Luego, utiliza el método \sphinxtitleref{deiconify()} en el objeto \sphinxtitleref{menu\_principal} para restaurar la
visibilidad y el enfoque en la ventana principal del programa.
\begin{quote}\begin{description}
\sphinxlineitem{Devuelve}
\sphinxAtStartPar
None

\end{description}\end{quote}

\end{fulllineitems}

\index{repetir\_si() (método de juego.Juego)@\spxentry{repetir\_si()}\spxextra{método de juego.Juego}}

\begin{fulllineitems}
\phantomsection\label{\detokenize{juego:juego.Juego.repetir_si}}
\pysigstartsignatures
\pysiglinewithargsret{\sphinxbfcode{\sphinxupquote{repetir\_si}}}{\sphinxparam{\DUrole{n}{letras}}\sphinxparamcomma \sphinxparam{\DUrole{n}{alfabeto}}\sphinxparamcomma \sphinxparam{\DUrole{n}{menu\_principal}}\sphinxparamcomma \sphinxparam{\DUrole{n}{modo}}}{}
\pysigstopsignatures
\sphinxAtStartPar
Reinicia el juego para continuar jugando.

\sphinxAtStartPar
Esta función reinicia el juego para continuar jugando con los mismos parámetros que
se utilizaron en la partida anterior. Puede reiniciar el juego en modo japonés\sphinxhyphen{}español
o español\sphinxhyphen{}japonés, según el modo en que se estaba jugando.

\sphinxAtStartPar
Parámetros:
:param letras: El diccionario de letras o caracteres a utilizar en el juego.
:param alfabeto: El alfabeto elegido para el juego (hiragana o katakana).
:param menu\_principal: Objeto Tkinter de la ventana principal del programa.
:param modo: Indicador del modo de juego (True para japonés\sphinxhyphen{}español, False para español\sphinxhyphen{}japonés).

\sphinxAtStartPar
Detalles:
La función destruye la ventana actual del juego utilizando el método \sphinxtitleref{destroy()}, lo que
cierra la ventana emergente del juego actual.

\sphinxAtStartPar
Luego, si el modo es japonés\sphinxhyphen{}español (\sphinxtitleref{True}), llama a la función \sphinxtitleref{japones\_a\_espanol} para
iniciar un nuevo juego en modo japonés\sphinxhyphen{}español. Si el modo es español\sphinxhyphen{}japonés (\sphinxtitleref{False}),
llama a la función \sphinxtitleref{espanol\_a\_japones} para iniciar un nuevo juego en modo español\sphinxhyphen{}japonés.
\begin{quote}\begin{description}
\sphinxlineitem{Devuelve}
\sphinxAtStartPar
None

\end{description}\end{quote}

\end{fulllineitems}


\end{fulllineitems}

\index{main() (en el módulo juego)@\spxentry{main()}\spxextra{en el módulo juego}}

\begin{fulllineitems}
\phantomsection\label{\detokenize{juego:juego.main}}
\pysigstartsignatures
\pysiglinewithargsret{\sphinxcode{\sphinxupquote{juego.}}\sphinxbfcode{\sphinxupquote{main}}}{\sphinxparam{\DUrole{n}{letras}}\sphinxparamcomma \sphinxparam{\DUrole{n}{modo\_juego}}\sphinxparamcomma \sphinxparam{\DUrole{n}{alfabeto}}\sphinxparamcomma \sphinxparam{\DUrole{n}{menu\_principal}}}{}
\pysigstopsignatures
\sphinxAtStartPar
Función principal que inicia el juego.

\sphinxAtStartPar
Esta función crea una instancia de la clase Juego y la utiliza para iniciar
y controlar el flujo del juego. El juego se juega de acuerdo al modo especificado
(japonés a español o español a japonés) y utiliza las letras y el alfabeto
proporcionados.

\sphinxAtStartPar
Parámetros:
:param letras: Un diccionario que contiene las letras o caracteres del juego.
:param modo\_juego: Un valor booleano que indica el modo de juego (japonés a español o español a japonés).
:param alfabeto: El alfabeto elegido para el juego (hiragana o katakana).
:param menu\_principal: Objeto Tkinter de la ventana principal del programa.

\sphinxAtStartPar
Detalles:
La función crea una instancia de la clase Juego llamada “juego” y luego invoca el método
“jugar” de la instancia creada. Este método controlará el flujo del juego y lo jugará según
el modo y los datos proporcionados.
\begin{quote}\begin{description}
\sphinxlineitem{Devuelve}
\sphinxAtStartPar
None

\end{description}\end{quote}

\end{fulllineitems}


\sphinxstepscope


\chapter{main module}
\label{\detokenize{main:module-main}}\label{\detokenize{main:main-module}}\label{\detokenize{main::doc}}\index{module@\spxentry{module}!main@\spxentry{main}}\index{main@\spxentry{main}!module@\spxentry{module}}\index{MenuPrincipal (clase en main)@\spxentry{MenuPrincipal}\spxextra{clase en main}}

\begin{fulllineitems}
\phantomsection\label{\detokenize{main:main.MenuPrincipal}}
\pysigstartsignatures
\pysiglinewithargsret{\sphinxbfcode{\sphinxupquote{class\DUrole{w}{ }}}\sphinxcode{\sphinxupquote{main.}}\sphinxbfcode{\sphinxupquote{MenuPrincipal}}}{\sphinxparam{\DUrole{n}{root}}}{}
\pysigstopsignatures
\sphinxAtStartPar
Bases: \sphinxcode{\sphinxupquote{object}}

\sphinxAtStartPar
Clase que representa la ventana principal del menú de la aplicación.

\sphinxAtStartPar
Esta clase se encarga de crear y gestionar la interfaz del menú principal de la aplicación. Proporciona opciones
para seleccionar modos de juego, mostrar el registro de progreso del jugador y acceder a la configuración de la
aplicación.
\begin{description}
\sphinxlineitem{Atributos:}
\sphinxAtStartPar
root (Tk): La ventana principal de la aplicación.
imagen\_hiragana (PhotoImage): Imagen para el botón de selección de modo Hiragana.
imagen\_katakana (PhotoImage): Imagen para el botón de selección de modo Katakana.
imagen\_registro (PhotoImage): Imagen para el botón de mostrar registro.
configuracion (PhotoImage): Imagen para el botón de acceso a configuración.

\sphinxlineitem{Métodos:}
\sphinxAtStartPar
\_\_init\_\_(self, root): Inicializa la ventana principal, carga imágenes y crea los botones.
decision\_hiragana(self): Abre el menú de selección de modo Hiragana.
decision\_katakana(self): Abre el menú de selección de modo Katakana.
mostrar\_registro(self): Muestra un gráfico de progreso del jugador a lo largo del tiempo.
abrir\_configuracion(self): Abre la ventana de configuración.
crear\_botones(self): Crea los botones en la ventana principal.

\end{description}
\index{abrir\_configuracion() (método de main.MenuPrincipal)@\spxentry{abrir\_configuracion()}\spxextra{método de main.MenuPrincipal}}

\begin{fulllineitems}
\phantomsection\label{\detokenize{main:main.MenuPrincipal.abrir_configuracion}}
\pysigstartsignatures
\pysiglinewithargsret{\sphinxbfcode{\sphinxupquote{abrir\_configuracion}}}{}{}
\pysigstopsignatures
\sphinxAtStartPar
Oculta la ventana principal y abre la ventana de configuración.

\sphinxAtStartPar
Detalles:
Esta función oculta la ventana principal actual utilizando el método “withdraw” y luego llama a la
función “main” del módulo “configuracion” para abrir la ventana de configuración. La instancia actual de la
ventana principal, “self.root”, se pasa a la función “main” para asegurarse de que la configuración se
realice en la misma ventana principal después de ocultarla.
\begin{quote}\begin{description}
\sphinxlineitem{Devuelve}
\sphinxAtStartPar
None

\end{description}\end{quote}

\end{fulllineitems}

\index{crear\_botones() (método de main.MenuPrincipal)@\spxentry{crear\_botones()}\spxextra{método de main.MenuPrincipal}}

\begin{fulllineitems}
\phantomsection\label{\detokenize{main:main.MenuPrincipal.crear_botones}}
\pysigstartsignatures
\pysiglinewithargsret{\sphinxbfcode{\sphinxupquote{crear\_botones}}}{}{}
\pysigstopsignatures
\sphinxAtStartPar
Crea los botones para seleccionar modos de juego y acceder a funciones adicionales.

\sphinxAtStartPar
Esta función crea varios botones en la ventana principal para interactuar con el programa. Los botones
incluyen opciones para seleccionar el alfabeto para el juego (Hiragana o Katakana), mostrar el registro de
progreso del jugador y acceder a la configuración de la aplicación.

\sphinxAtStartPar
Detalles:
Esta función utiliza la librería Tkinter para crear y posicionar los botones en la ventana principal. Cada
botón está asociado a una función específica (como “decision\_hiragana”, “decision\_katakana”, “mostrar\_registro”,
y “abrir\_configuracion”) y se configura para llamar a la función correspondiente cuando se hace clic en él.
\begin{quote}\begin{description}
\sphinxlineitem{Devuelve}
\sphinxAtStartPar
None

\end{description}\end{quote}

\end{fulllineitems}

\index{decision\_hiragana() (método de main.MenuPrincipal)@\spxentry{decision\_hiragana()}\spxextra{método de main.MenuPrincipal}}

\begin{fulllineitems}
\phantomsection\label{\detokenize{main:main.MenuPrincipal.decision_hiragana}}
\pysigstartsignatures
\pysiglinewithargsret{\sphinxbfcode{\sphinxupquote{decision\_hiragana}}}{}{}
\pysigstopsignatures
\sphinxAtStartPar
Oculta la ventana principal y llama a la función de menú del juego para elegir el modo de juego.

\sphinxAtStartPar
Detalles:
Esta función se activa al hacer click en el boton hiragana. Y oculta la ventana principal y abre el
menú de selección de modo de juego. Se le pasa un parámetro booleano para indicar si el modo de juego
seleccionado es Hiragana (True).
\begin{quote}\begin{description}
\sphinxlineitem{Devuelve}
\sphinxAtStartPar
None

\end{description}\end{quote}

\end{fulllineitems}

\index{decision\_katakana() (método de main.MenuPrincipal)@\spxentry{decision\_katakana()}\spxextra{método de main.MenuPrincipal}}

\begin{fulllineitems}
\phantomsection\label{\detokenize{main:main.MenuPrincipal.decision_katakana}}
\pysigstartsignatures
\pysiglinewithargsret{\sphinxbfcode{\sphinxupquote{decision\_katakana}}}{}{}
\pysigstopsignatures
\sphinxAtStartPar
Oculta la ventana principal y llama a la función de menú del juego para elegir el modo de juego.

\sphinxAtStartPar
Detalles:
Esta función se activa al hacer click en el boton katakana. Y oculta la ventana principal y abre el
menú de selección de modo de juego. Se le pasa un parámetro booleano para indicar si el modo de juego
seleccionado es Katakana (False).
\begin{quote}\begin{description}
\sphinxlineitem{Devuelve}
\sphinxAtStartPar
None

\end{description}\end{quote}

\end{fulllineitems}

\index{mostrar\_registro() (método de main.MenuPrincipal)@\spxentry{mostrar\_registro()}\spxextra{método de main.MenuPrincipal}}

\begin{fulllineitems}
\phantomsection\label{\detokenize{main:main.MenuPrincipal.mostrar_registro}}
\pysigstartsignatures
\pysiglinewithargsret{\sphinxbfcode{\sphinxupquote{mostrar\_registro}}}{}{}
\pysigstopsignatures
\sphinxAtStartPar
Oculta la ventana principal y muestra un gráfico del progreso del jugador a lo largo del tiempo
utilizando la función “mostrar\_grafico\_tkinter” de la clase “GestionDatos”.

\sphinxAtStartPar
Detalles:
Esta función utiliza la clase “GestionDatos” para mostrar un gráfico del progreso del jugador a lo
largo del tiempo en una ventana secundaria utilizando la librería Tkinter. Primero, oculta la ventana
principal actual y luego crea una instancia de la clase “GestionDatos” para acceder a la función
“mostrar\_grafico\_tkinter”. Dicha función es responsable de mostrar el gráfico en una ventana secundaria. El
parámetro “self.root” se pasa a la función “mostrar\_grafico\_tkinter” para asegurarse de que el gráfico se
muestre en la misma ventana principal después de ocultarla.
\begin{quote}\begin{description}
\sphinxlineitem{Devuelve}
\sphinxAtStartPar
None

\end{description}\end{quote}

\end{fulllineitems}


\end{fulllineitems}

\index{main() (en el módulo main)@\spxentry{main()}\spxextra{en el módulo main}}

\begin{fulllineitems}
\phantomsection\label{\detokenize{main:main.main}}
\pysigstartsignatures
\pysiglinewithargsret{\sphinxcode{\sphinxupquote{main.}}\sphinxbfcode{\sphinxupquote{main}}}{}{}
\pysigstopsignatures
\sphinxAtStartPar
Inicia la aplicación del menú principal.

\sphinxAtStartPar
Esta función es el punto de entrada de la aplicación. Crea una ventana principal de Tkinter, instancia la clase
“MenuPrincipal” para gestionar el menú y luego inicia el bucle principal de la aplicación, que se encarga de
mostrar la interfaz y responder a eventos del usuario.

\sphinxAtStartPar
Detalles:
Esta función crea la ventana principal utilizando la librería Tkinter y luego crea una instancia de la clase
“MenuPrincipal”, que se encarga de gestionar la interfaz del menú. Después de configurar todo, inicia el bucle
principal de la aplicación llamando al método “mainloop” de la ventana. Esto permite que la aplicación se ejecute
y responda a eventos de manera interactiva.
\begin{quote}\begin{description}
\sphinxlineitem{Devuelve}
\sphinxAtStartPar
None

\end{description}\end{quote}

\end{fulllineitems}


\sphinxstepscope


\chapter{menu\_juego module}
\label{\detokenize{menu_juego:module-menu_juego}}\label{\detokenize{menu_juego:menu-juego-module}}\label{\detokenize{menu_juego::doc}}\index{module@\spxentry{module}!menu\_juego@\spxentry{menu\_juego}}\index{menu\_juego@\spxentry{menu\_juego}!module@\spxentry{module}}\index{CaracteresSelector (clase en menu\_juego)@\spxentry{CaracteresSelector}\spxextra{clase en menu\_juego}}

\begin{fulllineitems}
\phantomsection\label{\detokenize{menu_juego:menu_juego.CaracteresSelector}}
\pysigstartsignatures
\pysiglinewithargsret{\sphinxbfcode{\sphinxupquote{class\DUrole{w}{ }}}\sphinxcode{\sphinxupquote{menu\_juego.}}\sphinxbfcode{\sphinxupquote{CaracteresSelector}}}{\sphinxparam{\DUrole{n}{decision\_alfabeto}}\sphinxparamcomma \sphinxparam{\DUrole{n}{decision\_modo}}\sphinxparamcomma \sphinxparam{\DUrole{n}{menu\_principal}}}{}
\pysigstopsignatures
\sphinxAtStartPar
Bases: \sphinxcode{\sphinxupquote{object}}

\sphinxAtStartPar
Clase que representa la ventana de selección de caracteres del juego.

\sphinxAtStartPar
Esta clase se encarga de manejar la ventana de selección de caracteres del juego, permitiendo al jugador
elegir diferentes conjuntos de caracteres para practicar.

\sphinxAtStartPar
Parámetros:
:param decision\_alfabeto (bool): Indica si se usará el alfabeto de hiragana (True) o katakana (False).
:param decision\_modo (bool): Indica el modo de juego: Japonés a Español (True) o Español a Japonés (False).
:param menu\_principal: La ventana principal del menú.
\begin{description}
\sphinxlineitem{Attributos:}
\sphinxAtStartPar
opcion\_vocales (bool): Indica si la opción de caracteres de vocales está seleccionada.
opcion\_basico (bool): Indica si la opción de caracteres básicos está seleccionada.
opcion\_compuesto (bool): Indica si la opción de caracteres compuestos está seleccionada.
opcion\_combinado1 (bool): Indica si la opción de caracteres combinados tipo 1 está seleccionada.
opcion\_combinado2 (bool): Indica si la opción de caracteres combinados tipo 2 está seleccionada.
opcion\_vocales\_var: Variable booleana asociada a la opción de vocales.
opcion\_basico\_var: Variable booleana asociada a la opción de caracteres básicos.
opcion\_compuesto\_var: Variable booleana asociada a la opción de caracteres compuestos.
opcion\_combinado1\_var: Variable booleana asociada a la opción de caracteres combinados tipo 1.
opcion\_combinado2\_var: Variable booleana asociada a la opción de caracteres combinados tipo 2.
menu\_principal: La ventana principal del menú.
decisionAlfabeto (bool): La decisión sobre qué alfabeto se usará.
decisionModo (bool): La decisión sobre el modo de juego.
caracteres (dict): Un diccionario que almacenará los caracteres seleccionados.

\sphinxlineitem{Métodos:}
\sphinxAtStartPar
\_\_init\_\_(decision\_alfabeto, decision\_modo, menu\_principal): Inicializa la clase
toggle\_opcion(opcion): Cambia el estado de una opción de caracteres.
continuar(): Oculta la ventana actual y abre la ventana de selección de caracteres.
crear\_interfaz(alfabeto\_usado): Crea la interfaz gráfica para la selección de opciones de caracteres.
seleccionar\_caracteres(): Procesa las opciones seleccionadas y almacena los caracteres correspondientes.
iniciar\_juego(): Inicia el juego con los caracteres seleccionados.

\end{description}
\index{continuar() (método de menu\_juego.CaracteresSelector)@\spxentry{continuar()}\spxextra{método de menu\_juego.CaracteresSelector}}

\begin{fulllineitems}
\phantomsection\label{\detokenize{menu_juego:menu_juego.CaracteresSelector.continuar}}
\pysigstartsignatures
\pysiglinewithargsret{\sphinxbfcode{\sphinxupquote{continuar}}}{}{}
\pysigstopsignatures
\sphinxAtStartPar
Oculta la ventana actual y procede a la selección de caracteres.

\sphinxAtStartPar
Detalles:
Esta función se activa cuando el botón «Continuar» en la ventana de selección de caracteres es presionado.
Su propósito es ocultar la ventana actual (ventana de selección de caracteres) utilizando el método \sphinxtitleref{withdraw()}
y luego proceder a la selección de caracteres invocando la función \sphinxtitleref{seleccionar\_caracteres()}.

\sphinxAtStartPar
La función \sphinxtitleref{seleccionar\_caracteres()} es responsable de determinar qué opciones de caracteres fueron
seleccionadas y construir el conjunto final de caracteres a utilizar en el juego.
\begin{quote}\begin{description}
\sphinxlineitem{Devuelve}
\sphinxAtStartPar
None

\end{description}\end{quote}

\end{fulllineitems}

\index{crear\_interfaz() (método de menu\_juego.CaracteresSelector)@\spxentry{crear\_interfaz()}\spxextra{método de menu\_juego.CaracteresSelector}}

\begin{fulllineitems}
\phantomsection\label{\detokenize{menu_juego:menu_juego.CaracteresSelector.crear_interfaz}}
\pysigstartsignatures
\pysiglinewithargsret{\sphinxbfcode{\sphinxupquote{crear\_interfaz}}}{\sphinxparam{\DUrole{n}{alfabeto\_usado}}}{}
\pysigstopsignatures
\sphinxAtStartPar
Crea la interfaz gráfica de la ventana de selección de opciones de caracteres.

\sphinxAtStartPar
Parámetros:
:param alfabeto\_usado: Indica si se usará el alfabeto de hiragana (True) o katakana (False).

\sphinxAtStartPar
Detalles:
Esta función es responsable de crear la interfaz gráfica de la ventana de selección de opciones de
caracteres. Utiliza la biblioteca tkinter para crear los elementos gráficos, como los botones de selección y
los checkbuttons. También se encarga de establecer los comandos asociados a los checkbuttons y de mostrar
tooltips informativos cuando el cursor del mouse pasa sobre ellos.

\sphinxAtStartPar
Después de construir la interfaz gráfica, establece las dimensiones y la posición de la ventana mediante el
método \sphinxtitleref{geometry()} y empaqueta los elementos en la ventana. Finalmente, muestra la ventana con las opciones de
selección de caracteres.
\begin{quote}\begin{description}
\sphinxlineitem{Devuelve}
\sphinxAtStartPar
None

\end{description}\end{quote}

\end{fulllineitems}

\index{iniciar\_juego() (método de menu\_juego.CaracteresSelector)@\spxentry{iniciar\_juego()}\spxextra{método de menu\_juego.CaracteresSelector}}

\begin{fulllineitems}
\phantomsection\label{\detokenize{menu_juego:menu_juego.CaracteresSelector.iniciar_juego}}
\pysigstartsignatures
\pysiglinewithargsret{\sphinxbfcode{\sphinxupquote{iniciar\_juego}}}{}{}
\pysigstopsignatures
\sphinxAtStartPar
Inicia el juego con las configuraciones y caracteres seleccionados.

\sphinxAtStartPar
Detalles: Esta función cierra la ventana actual de selección de caracteres y llama a la función “main” del
módulo “juego” para iniciar el juego con las configuraciones y caracteres seleccionados. Se pasan como
argumentos el diccionario de caracteres seleccionados, la decisión sobre el modo de juego y la decisión sobre
el alfabeto a utilizar.
\begin{quote}\begin{description}
\sphinxlineitem{Devuelve}
\sphinxAtStartPar
None

\end{description}\end{quote}

\end{fulllineitems}

\index{seleccionar\_caracteres() (método de menu\_juego.CaracteresSelector)@\spxentry{seleccionar\_caracteres()}\spxextra{método de menu\_juego.CaracteresSelector}}

\begin{fulllineitems}
\phantomsection\label{\detokenize{menu_juego:menu_juego.CaracteresSelector.seleccionar_caracteres}}
\pysigstartsignatures
\pysiglinewithargsret{\sphinxbfcode{\sphinxupquote{seleccionar\_caracteres}}}{}{}
\pysigstopsignatures
\sphinxAtStartPar
Selecciona los caracteres basados en las opciones elegidas por el usuario.

\sphinxAtStartPar
Detalles: Esta función determina qué caracteres se incluirán en el juego en función de las opciones
seleccionadas por el usuario en la ventana de selección de caracteres. Primero, verifica si se va a utilizar
el alfabeto de hiragana o katakana según la decisión de alfabeto.

\sphinxAtStartPar
Luego, utiliza las variables de opción (opcion\_vocales, opcion\_basico, etc.) para decidir qué conjuntos de
caracteres se deben agregar al diccionario “caracteres”. Cada conjunto se agrega utilizando la operación de
unión (|) para combinar los conjuntos de caracteres seleccionados.

\sphinxAtStartPar
Una vez que se han seleccionado los caracteres, se llama a la función “iniciar\_juego()” para comenzar el
juego con los caracteres seleccionados.
\begin{quote}\begin{description}
\sphinxlineitem{Devuelve}
\sphinxAtStartPar
None

\end{description}\end{quote}

\end{fulllineitems}

\index{toggle\_opcion() (método de menu\_juego.CaracteresSelector)@\spxentry{toggle\_opcion()}\spxextra{método de menu\_juego.CaracteresSelector}}

\begin{fulllineitems}
\phantomsection\label{\detokenize{menu_juego:menu_juego.CaracteresSelector.toggle_opcion}}
\pysigstartsignatures
\pysiglinewithargsret{\sphinxbfcode{\sphinxupquote{toggle\_opcion}}}{\sphinxparam{\DUrole{n}{opcion}}}{}
\pysigstopsignatures
\sphinxAtStartPar
Cambia el estado de una opción de caracteres.

\sphinxAtStartPar
Parámetros:
:param opcion: (str) La opción de caracteres cuyo estado se cambiará.

\sphinxAtStartPar
Detalles:
Esta función se utiliza para cambiar el estado de una opción de caracteres. Recibe como argumento una cadena que
representa la opción («Vocales», «Basico», etc.) y cambia el valor correspondiente en las variables de estado
(\sphinxtitleref{opcion\_vocales}, \sphinxtitleref{opcion\_basico}, etc.).

\sphinxAtStartPar
Si la opción estaba en \sphinxtitleref{True}, cambiará a \sphinxtitleref{False}, y viceversa. Esto permite al usuario seleccionar y
deseleccionar las opciones según sea necesario para el juego.
\begin{quote}\begin{description}
\sphinxlineitem{Devuelve}
\sphinxAtStartPar
None

\end{description}\end{quote}

\end{fulllineitems}


\end{fulllineitems}

\index{MenuJuego (clase en menu\_juego)@\spxentry{MenuJuego}\spxextra{clase en menu\_juego}}

\begin{fulllineitems}
\phantomsection\label{\detokenize{menu_juego:menu_juego.MenuJuego}}
\pysigstartsignatures
\pysiglinewithargsret{\sphinxbfcode{\sphinxupquote{class\DUrole{w}{ }}}\sphinxcode{\sphinxupquote{menu\_juego.}}\sphinxbfcode{\sphinxupquote{MenuJuego}}}{\sphinxparam{\DUrole{n}{alfabeto\_usado}}\sphinxparamcomma \sphinxparam{\DUrole{n}{menu\_principal}}}{}
\pysigstopsignatures
\sphinxAtStartPar
Bases: \sphinxcode{\sphinxupquote{object}}

\sphinxAtStartPar
Clase que representa el menú principal del juego.

\sphinxAtStartPar
Esta clase crea una ventana de menú donde el jugador puede elegir entre los modos
de juego «Japonés a Español» o «Español a Japonés».

\sphinxAtStartPar
Parámetros:
:param alfabeto\_usado: Booleano que indica si se usará el alfabeto hiragana (True) o katakana (False).
:param menu\_principal: Objeto Tkinter de la ventana principal del programa.

\sphinxAtStartPar
Métodos:
\sphinxhyphen{} \_\_init\_\_(self, alfabeto\_usado, menu\_principal): Constructor que inicializa la ventana y otros atributos.
\sphinxhyphen{} japones\_a\_espanol(self): Método que inicia el juego en el modo «Japonés a Español».
\sphinxhyphen{} espanol\_a\_japones(self): Método que inicia el juego en el modo «Español a Japonés».
\sphinxhyphen{} crear\_interfaz(self): Método que crea la interfaz gráfica del menú.
\index{crear\_interfaz() (método de menu\_juego.MenuJuego)@\spxentry{crear\_interfaz()}\spxextra{método de menu\_juego.MenuJuego}}

\begin{fulllineitems}
\phantomsection\label{\detokenize{menu_juego:menu_juego.MenuJuego.crear_interfaz}}
\pysigstartsignatures
\pysiglinewithargsret{\sphinxbfcode{\sphinxupquote{crear\_interfaz}}}{}{}
\pysigstopsignatures
\sphinxAtStartPar
Crea y muestra la interfaz gráfica para la selección de modo de juego.

\sphinxAtStartPar
Esta función se encarga de crear y mostrar la interfaz gráfica en la ventana de selección de modo de juego.
La interfaz incluye dos botones con imágenes correspondientes a los modos de juego disponibles: «Japonés a
Español» y «Español a Japonés».

\sphinxAtStartPar
Detalles:
La función crea dos botones de la clase \sphinxtitleref{tk.Button} para los modos de juego «Japonés a Español» y
«Español a Japonés». Cada botón tiene una imagen asociada (\sphinxtitleref{self.imagen\_japones\_a\_espanol} o
\sphinxtitleref{self.imagen\_espanol\_a\_japones}), un comando que se activará cuando se haga clic en el botón y dimensiones de
ancho y alto.

\sphinxAtStartPar
Los botones se posicionan en la ventana utilizando el método \sphinxtitleref{place()} y las coordenadas \sphinxtitleref{(x, y)} especificadas.
El botón «Japonés a Español» se coloca en la posición \sphinxtitleref{(65, 50)} y el botón «Español a Japonés» se coloca en la
posición \sphinxtitleref{(500, 50)}.

\sphinxAtStartPar
Cuando se hace clic en cualquiera de los botones, se ejecutará el comando correspondiente (
\sphinxtitleref{self.japones\_a\_espanol} o \sphinxtitleref{self.espanol\_a\_japones}) para cambiar el modo de juego y abrir la ventana de
selección de caracteres.
\begin{quote}\begin{description}
\sphinxlineitem{Devuelve}
\sphinxAtStartPar
None

\end{description}\end{quote}

\end{fulllineitems}

\index{espanol\_a\_japones() (método de menu\_juego.MenuJuego)@\spxentry{espanol\_a\_japones()}\spxextra{método de menu\_juego.MenuJuego}}

\begin{fulllineitems}
\phantomsection\label{\detokenize{menu_juego:menu_juego.MenuJuego.espanol_a_japones}}
\pysigstartsignatures
\pysiglinewithargsret{\sphinxbfcode{\sphinxupquote{espanol\_a\_japones}}}{}{}
\pysigstopsignatures
\sphinxAtStartPar
Cambia la decisión de modo a «Español a Japonés» y abre la ventana de selección de caracteres.

\sphinxAtStartPar
Esta función se activa cuando el jugador elige el modo «Español a Japonés» en el menú de selección de modo.
Cambia el atributo \sphinxtitleref{decision} de la clase a \sphinxtitleref{False}, indicando que se ha seleccionado el modo «Español a
Japonés».

\sphinxAtStartPar
Detalles:
La función establece \sphinxtitleref{self.decision} en \sphinxtitleref{False} para indicar que el modo de juego seleccionado es
«Español a Japonés». Luego, utiliza el método \sphinxtitleref{destroy()} en \sphinxtitleref{self.menu} para cerrar la ventana actual del
menú.

\sphinxAtStartPar
Se crea una instancia de la clase \sphinxtitleref{CaracteresSelector}, pasando los parámetros \sphinxtitleref{self.alfabeto} (decisión del
alfabeto), \sphinxtitleref{self.decision} (decisión de modo) y \sphinxtitleref{self.menu\_principal} (ventana principal del programa). Esto
abre la ventana de selección de caracteres para que el jugador elija qué caracteres utilizar en el juego.
\begin{quote}\begin{description}
\sphinxlineitem{Devuelve}
\sphinxAtStartPar
None

\end{description}\end{quote}

\end{fulllineitems}

\index{japones\_a\_espanol() (método de menu\_juego.MenuJuego)@\spxentry{japones\_a\_espanol()}\spxextra{método de menu\_juego.MenuJuego}}

\begin{fulllineitems}
\phantomsection\label{\detokenize{menu_juego:menu_juego.MenuJuego.japones_a_espanol}}
\pysigstartsignatures
\pysiglinewithargsret{\sphinxbfcode{\sphinxupquote{japones\_a\_espanol}}}{}{}
\pysigstopsignatures
\sphinxAtStartPar
Cambia la decisión de modo a «Japonés a Español» y abre la ventana de selección de caracteres.

\sphinxAtStartPar
Esta función se activa cuando el jugador elige el modo «Japonés a Español» en el menú de selección de modo.
Cambia el atributo \sphinxtitleref{decision} de la clase a \sphinxtitleref{True}, indicando que se ha seleccionado el modo «Japonés a
Español».

\sphinxAtStartPar
Detalles:
La función establece \sphinxtitleref{self.decision} en \sphinxtitleref{True} para indicar que el modo de juego seleccionado es
«Japonés a Español». Luego, utiliza el método \sphinxtitleref{destroy()} en \sphinxtitleref{self.menu} para cerrar la ventana actual del
menú.

\sphinxAtStartPar
Se crea una instancia de la clase \sphinxtitleref{CaracteresSelector}, pasando los parámetros \sphinxtitleref{self.alfabeto} (decisión del
alfabeto), \sphinxtitleref{self.decision} (decisión de modo) y \sphinxtitleref{self.menu\_principal} (ventana principal del programa). Esto
abre la ventana de selección de caracteres para que el jugador elija qué caracteres utilizar en el juego.
\begin{quote}\begin{description}
\sphinxlineitem{Devuelve}
\sphinxAtStartPar
None

\end{description}\end{quote}

\end{fulllineitems}


\end{fulllineitems}

\index{cerrar\_ventana() (en el módulo menu\_juego)@\spxentry{cerrar\_ventana()}\spxextra{en el módulo menu\_juego}}

\begin{fulllineitems}
\phantomsection\label{\detokenize{menu_juego:menu_juego.cerrar_ventana}}
\pysigstartsignatures
\pysiglinewithargsret{\sphinxcode{\sphinxupquote{menu\_juego.}}\sphinxbfcode{\sphinxupquote{cerrar\_ventana}}}{\sphinxparam{\DUrole{n}{menu}}\sphinxparamcomma \sphinxparam{\DUrole{n}{menu\_principal}}}{}
\pysigstopsignatures
\sphinxAtStartPar
Cierra una ventana secundaria y restaura la visibilidad de la ventana principal.

\sphinxAtStartPar
Parámetros:
:param menu: La ventana secundaria a cerrar.
:param menu\_principal: La ventana principal que se restaurará a la visibilidad.

\sphinxAtStartPar
Detalles:
Esta función oculta la ventana secundaria (“menu”) utilizando el método “withdraw()”, luego la destruye con el
método “destroy()”. Posteriormente, hace visible la ventana principal (“menu\_principal”) utilizando el método
“deiconify()”, lo que permite que la ventana principal sea nuevamente visible para el usuario.
\begin{quote}\begin{description}
\sphinxlineitem{Devuelve}
\sphinxAtStartPar
None

\end{description}\end{quote}

\end{fulllineitems}

\index{main() (en el módulo menu\_juego)@\spxentry{main()}\spxextra{en el módulo menu\_juego}}

\begin{fulllineitems}
\phantomsection\label{\detokenize{menu_juego:menu_juego.main}}
\pysigstartsignatures
\pysiglinewithargsret{\sphinxcode{\sphinxupquote{menu\_juego.}}\sphinxbfcode{\sphinxupquote{main}}}{\sphinxparam{\DUrole{n}{alfabeto\_usado}}\sphinxparamcomma \sphinxparam{\DUrole{n}{menu\_principal}}}{}
\pysigstopsignatures
\sphinxAtStartPar
Función principal para iniciar la aplicación del juego.

\sphinxAtStartPar
Detalles:
Esta función es el punto de entrada para iniciar la aplicación del juego. Toma dos argumentos: “alfabeto\_usado”
que indica si se utilizará el alfabeto de hiragana o katakana, y “menu\_principal” que es la ventana principal
del menú.

\sphinxAtStartPar
Parámetros:
:param alfabeto\_usado: Un valor booleano que indica si se usará el alfabeto de hiragana (True) o katakana (False).
:param menu\_principal: La ventana principal del menú.

\sphinxAtStartPar
La función crea una instancia de la clase “MenuJuego”, pasando los valores de “alfabeto\_usado” y “menu\_principal”
como argumentos. Esto inicia la ventana del menú del juego.
\begin{quote}\begin{description}
\sphinxlineitem{Devuelve}
\sphinxAtStartPar
None

\end{description}\end{quote}

\end{fulllineitems}


\sphinxstepscope


\chapter{documentacion}
\label{\detokenize{modules:documentacion}}\label{\detokenize{modules::doc}}
\sphinxstepscope


\chapter{tooltip module}
\label{\detokenize{tooltip:module-tooltip}}\label{\detokenize{tooltip:tooltip-module}}\label{\detokenize{tooltip::doc}}\index{module@\spxentry{module}!tooltip@\spxentry{tooltip}}\index{tooltip@\spxentry{tooltip}!module@\spxentry{module}}\index{Tooltip (clase en tooltip)@\spxentry{Tooltip}\spxextra{clase en tooltip}}

\begin{fulllineitems}
\phantomsection\label{\detokenize{tooltip:tooltip.Tooltip}}
\pysigstartsignatures
\pysigline{\sphinxbfcode{\sphinxupquote{class\DUrole{w}{ }}}\sphinxcode{\sphinxupquote{tooltip.}}\sphinxbfcode{\sphinxupquote{Tooltip}}}
\pysigstopsignatures
\sphinxAtStartPar
Bases: \sphinxcode{\sphinxupquote{object}}

\sphinxAtStartPar
Clase que maneja la creación y visualización de tooltips (información emergente) en widgets.

\sphinxAtStartPar
Attributes:
\sphinxhyphen{} tooltip: Ventana emergente que muestra el tooltip.

\sphinxAtStartPar
Métodos:
\sphinxhyphen{} \_\_init\_\_(self): Inicializa una nueva instancia de Tooltip.
\sphinxhyphen{} mostrar\_tooltip(self, widget, text\_dict): Muestra un tooltip en un widget específico.
\sphinxhyphen{} ocultar\_tooltip(self): Oculta el tooltip si está visible.
\index{mostrar\_tooltip() (método de tooltip.Tooltip)@\spxentry{mostrar\_tooltip()}\spxextra{método de tooltip.Tooltip}}

\begin{fulllineitems}
\phantomsection\label{\detokenize{tooltip:tooltip.Tooltip.mostrar_tooltip}}
\pysigstartsignatures
\pysiglinewithargsret{\sphinxbfcode{\sphinxupquote{mostrar\_tooltip}}}{\sphinxparam{\DUrole{n}{widget}}\sphinxparamcomma \sphinxparam{\DUrole{n}{text\_dict}}}{}
\pysigstopsignatures
\sphinxAtStartPar
Muestra un tooltip en un widget específico con información proporcionada en text\_dict.

\sphinxAtStartPar
Parámetros:
:param widget: El widget en el que se mostrará el tooltip.
:param text\_dict: Un diccionario que contiene la información a mostrar en el tooltip.

\sphinxAtStartPar
Detalles:
La función calcula las coordenadas en las que se debe mostrar el tooltip cerca del widget.
Crea una nueva ventana emergente (Toplevel) y configura su apariencia y contenido utilizando
el contenido de text\_dict. El tooltip se ajusta automáticamente en tamaño para acomodar su contenido.
\begin{quote}\begin{description}
\sphinxlineitem{Devuelve}
\sphinxAtStartPar
None

\end{description}\end{quote}

\end{fulllineitems}

\index{ocultar\_tooltip() (método de tooltip.Tooltip)@\spxentry{ocultar\_tooltip()}\spxextra{método de tooltip.Tooltip}}

\begin{fulllineitems}
\phantomsection\label{\detokenize{tooltip:tooltip.Tooltip.ocultar_tooltip}}
\pysigstartsignatures
\pysiglinewithargsret{\sphinxbfcode{\sphinxupquote{ocultar\_tooltip}}}{}{}
\pysigstopsignatures
\sphinxAtStartPar
Oculta el tooltip si está visible.

\sphinxAtStartPar
Detalles:
La función verifica si hay un tooltip mostrándose en la ventana actual. Si es así,
se destruye la ventana del tooltip, ocultándolo.
\begin{quote}\begin{description}
\sphinxlineitem{Devuelve}
\sphinxAtStartPar
None

\end{description}\end{quote}

\end{fulllineitems}


\end{fulllineitems}



\chapter{Indices and tables}
\label{\detokenize{index:indices-and-tables}}\begin{itemize}
\item {} 
\sphinxAtStartPar
\DUrole{xref,std,std-ref}{genindex}

\item {} 
\sphinxAtStartPar
\DUrole{xref,std,std-ref}{modindex}

\item {} 
\sphinxAtStartPar
\DUrole{xref,std,std-ref}{search}

\end{itemize}


\renewcommand{\indexname}{Índice de Módulos Python}
\begin{sphinxtheindex}
\let\bigletter\sphinxstyleindexlettergroup
\bigletter{c}
\item\relax\sphinxstyleindexentry{configuracion}\sphinxstyleindexpageref{configuracion:\detokenize{module-configuracion}}
\item\relax\sphinxstyleindexentry{custom\_message\_box}\sphinxstyleindexpageref{custom_message_box:\detokenize{module-custom_message_box}}
\indexspace
\bigletter{d}
\item\relax\sphinxstyleindexentry{datos}\sphinxstyleindexpageref{datos:\detokenize{module-datos}}
\indexspace
\bigletter{j}
\item\relax\sphinxstyleindexentry{juego}\sphinxstyleindexpageref{juego:\detokenize{module-juego}}
\indexspace
\bigletter{m}
\item\relax\sphinxstyleindexentry{main}\sphinxstyleindexpageref{main:\detokenize{module-main}}
\item\relax\sphinxstyleindexentry{menu\_juego}\sphinxstyleindexpageref{menu_juego:\detokenize{module-menu_juego}}
\indexspace
\bigletter{t}
\item\relax\sphinxstyleindexentry{tooltip}\sphinxstyleindexpageref{tooltip:\detokenize{module-tooltip}}
\end{sphinxtheindex}

\renewcommand{\indexname}{Índice}
\printindex
\end{document}